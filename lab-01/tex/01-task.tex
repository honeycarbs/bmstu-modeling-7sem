\chapter{Задание}

\noindent Исследовать функцию и плотность распределения:
\begin{itemize}
	\item равномерного;
	\item нормального.
\end{itemize}
Разработать программу для построения графиков этих распределений. Реализовать графический интерфейс, позволяющий задавать параметры для построения графиков.

\chapter{Математическая формализация}
\section{Непрерывное равномерное распределение}
Говорят, что случайная величина $X$ имеет непрерывное равномерное распределение на отрезке $\displaystyle [a,b]$, если её  функция плотности имеет вид (\ref{eq:uniformpdf}):

\begin{equation}\label{eq:uniformpdf}
	 f_{X}(x) = \begin{cases}
	 				\cfrac{1}{b - a}, & x \in [a,b] \\
	 				0, & x \notin [a,b] \\
	 			\end{cases}.
\end{equation}
Обозначается: $X\sim U[a,b]$.

\noindent Функция распределения равномерной случайной величины  $X\sim U[a,b]$ (\ref{eq:uniformcdf}):
\begin{equation}\label{eq:uniformcdf}
	F_{X}(x)\equiv \mathbb {P} (X\leqslant x) = 
	\begin{cases}
		0, & x < a \\
		\cfrac{x - a}{b - a}, & a\leqslant x < b \\
		1, & x\geqslant b
	\end{cases}.
\end{equation}

\section{Нормальное распределение}
Говорят, что случайная величина $X$ имеет нормальное распределение с параметрами $\mu$ и $\sigma^2$ ($\sigma^2 > 0$), если её функция плотности имеет вид (\ref{eq:normpdf}):
\begin{equation}\label{eq:normpdf}
	f_{X}(x) = \cfrac{1}{\sigma\sqrt{2\pi}}e^{\cfrac{(x - \mu)^2}{2\sigma^2}}\;, x \in \mathbb{R}
\end{equation}
Обозначается: $X\sim N(\mu, \sigma^2)$.

Функция распределения нормальной случайной величины $X\sim N(\mu, \sigma^2)$:
\begin{equation}\label{eq:normcdf}
	\frac {1}{2}\left[1+\operatorname {erf} \left({\frac {x-\mu }{\sigma {\sqrt {2}}}}\right)\right],
\end{equation}
где
\begin{equation}\label{eq:erf}
	\operatorname {erf} (x)=\cfrac {2}{\sqrt {\pi}}\int_{0}^{x}e^{-t^2}dt.
\end{equation}

%{\displaystyle F(x)=\Phi \left({\frac {x-\mu }{\sigma }}\right)={.}