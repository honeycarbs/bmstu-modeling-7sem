\chapter{Программный код}

Для реализации программы был выбран язык Python. % :(

На листинге \ref{lst:dst} представлен программный код абстрактного класса Distribution.

\begin{lstlisting}[language=Python,
				   frame= tb,
				   numbers=left,
				   numberstyle=\footnotesize,
				   caption={Абстрактный класс Distribution},
				   label={lst:dst}]
from abc import ABCMeta, abstractmethod

class AbstractDistribution(object):
	__metaclass__ = ABCMeta

	@abstractmethod
	def CDF(self, x):
		pass

	@abstractmethod
	def PDF(self, x):
		pass
\end{lstlisting}

На листинге \ref{lst:norm} представлен программный код класса Gaussian (нормальное распределение), реализующий абстрактный класс Distribution.

\begin{lstlisting}[language=Python,
				   frame= tb,
				   numbers=left,
				   numberstyle=\footnotesize,
				   caption={Класс Gaussian},
				   label={lst:norm}]
from probability.distribution.distribution import AbstractDistribution

import math
from scipy import special


class Gaussian(AbstractDistribution):
	__slots__ = ["mean",
				 "deviation",
				 "variance"]

	def __init__(self, mean, variance):
		if variance <= 0.0:
			raise ValueError("Variance can't non-positive")

		self.mean = mean
		self.variance = variance
		self.deviation = math.sqrt(variance)

	def CDF(self, x) -> float:
		return 0.5 * special.erfc(-(x-self.mean)/(self.deviation*math.sqrt(2)))

	def PDF(self, x) -> float:
		m = self.deviation * math.sqrt(2*math.pi)
		e = math.exp(-math.pow(x-self.mean, 2) / (2 * self.variance))
		return e / m
\end{lstlisting}

На листинге \ref{lst:uniform} представлен программный код класса Uniform (равномерное распределение), реализующий абстрактный класс Distribution.

\begin{lstlisting}[language=Python,
	frame= tb,
	numbers=left,
	numberstyle=\footnotesize,
	caption={Класс Uniform},
	label={lst:uniform}]
from probability.distribution.distribution import AbstractDistribution
	
class Uniform(AbstractDistribution):
	__slots__ = ["min",
				 "max"]

	def __init__(self, min, max):
		self.min = min
		self.max = max


	def CDF(self, x) -> float:
		if x < self.min:
			return 0
		if x > self.max:
			return 1

	return (x - self.min) / (self.max - self.min)

	def PDF(self, x) -> float:
		if x < self.min:
			return 0
		if x > self.max:
			return 0

		return 1 / (self.max - self.min)
\end{lstlisting}

При запуске приложения на координатных плоскостях изображены графики со следующими параметрами: нормальное распределение -- коэффициенты стандартного нормального распределения ($\mu = 0.0, \; \sigma^2 = 1.0$), равномерное распределение -- $a=-2.0,\; b=2.0$. На листинге \ref{lst:gui} представлена часть программного кода построения графиков с коэффициентами по умолчанию. 

\begin{lstlisting}[language=Python,
	frame= tb,
	numbers=left,
	numberstyle=\footnotesize,
	caption={Часть программного кода построения графиков},
	label={lst:gui}]

figCDFGaussian = Figure(figsize=(5, 4), dpi=100)
figPDFGaussian = Figure(figsize=(5, 4), dpi=100)

figCDFUniform = Figure(figsize=(5, 4), dpi=100)
figPDFUniform = Figure(figsize=(5, 4), dpi=100)

xs = np.arange(-5, 5, .1)

u = Uniform(-2, 2)
ysPDFUniform = []
ysCDFUniform = []

g = Gaussian(0, 1)
ysPDFGaussian = []
ysCDFGaussian = []

for x in xs:
	ysPDFGaussian.append(g.PDF(x))
	ysCDFGaussian.append(g.CDF(x))

	ysPDFUniform.append(u.PDF(x))
	ysCDFUniform.append(u.CDF(x))

axCDFGaussian = figCDFGaussian.add_subplot()
lineCDFGaussian, = axCDFGaussian.plot(xs, ysCDFGaussian,
										 color='xkcd:mauve')

axPDFGaussian = figPDFGaussian.add_subplot()
linePDFGaussian, = axPDFGaussian.plot(xs, ysPDFGaussian,
									  color='xkcd:avocado')

canvasPDFGaussian = FigureCanvasTkAgg(figPDFGaussian,
									  master=root)
canvasPDFGaussian.draw()

canvasCDFGaussian = FigureCanvasTkAgg(figCDFGaussian,
									  master=root)
canvasCDFGaussian.draw()
	
\end{lstlisting}