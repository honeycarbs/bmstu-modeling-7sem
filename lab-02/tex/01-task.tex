\chapter{Задание}

Для сложной системы S, имеющей не более 10 состояний, определить время нахождения системы в
 предельных состояниях, то есть при установившемся режиме работы.
\chapter{Математическая формализация}
Для математической формализации функционирования устройств процесс в котором
развивается в форме случайного процесса может быть с успехом применен аппарат
из теории вероятностей (теории массового обсуживания) для так называемых 
марковских случайных процессов.

Случайный процесс протекающий в некоторой системе S, называют марковским если он
 обладает следующим свойством:   

\noindent\textit{для каждого момента времени вероятность любого состояния в будущем зависит 
только от его состояния в настоящем и не зависит от того, когда и каким образом 
система пришла в это состояние.}

Решив уравнение Колмогорова, определить вероятность нахождения системы в
этом состоянии. Уравнение Колмогорова в общем виде можно представить следующим образом (\ref{eq:kolm}):
\begin{equation}\label{eq:kolm}
    F = (p'(t), P(t), \Lambda) = 0,
\end{equation} 
Где $\Lambda$ — это коэффициенты.

\noindentУравнение Колмогорова строится по следующим правилам:
\begin{itemize}
    \item в левой части каждого уравнения стоит производная вероятности состояния, а правая часть содержит столько членов сколько стрелок связано с этим состоянием. Если стрелка направлена из
    состояния - знак минус, если в состояние - знак плюс;
    \item каждый член равен произведению плотности вероятности перехода (интенсивность) соответствующей данной стрелке, умноженной на вероятность того состояния из которого исходит стрелка.
\end{itemize}
Для получения предельных вероятностей, то есть вероятностей в стационарном 
режиме работы при $t \rightarrow \inf$, необходимо приравнять левые части 
уравнений к нулю. Получается система линейных уравнений. Необходимо 
соблюдение условия нормировки: (\ref{eq:norm}):
\begin{equation}\label{eq:norm}
    \sum_{i}^{n}p_i = 1
\end{equation}
