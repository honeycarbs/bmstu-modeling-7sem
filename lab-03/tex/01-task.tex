\chapter{Задание}
Написать программу, которая генерирует псевдослучайную последовательность 
одноразрядных, двухразрядных и трёхразрядных целых чисел с использование 
табличного и алгоритмического способа.

Для каждой сгенерированной последовательности чисел вычислить 
критерий оценки случайности.
\chapter{Математическая формализация}
Существует три способа получения последовательности случайных чисел:
\begin{itemize}
    \item аппаратный способ;
    \item табличный (файловый) способ;
    \item алгоритмический способ.
\end{itemize}

При использовании аппаратного способа случайная величина вырабатывается специальной электрической приставкой
(генератор случайных чисел) как правило внешнее устройство компьютера не требует
других устройств и операций кроме обращения к устройству.

Если случайные числа, оформленные в виде таблицы, помещать во внешнюю или 
оперативную память ЭВМ, предварительно сформировав из них соответствующий файл, 
то такой способ будет называться табличным. Однако, хранение файла во внешней 
памяти при частном обращении в процессе статистического моделирования не 
рационально, так как вызывает увеличение затрат машинного времени при 
моделировании системы из-за необходимости обращения к внешнему накопителю.

Алгоритмический способ – это способ получения последовательности 
случайных чисел, основанный на формировании случайных чисел в ЭВМ 
с использованием специальных алгоритмов и реализующих их программ. В качестве используемого 
метода генерации последовательности случайных чисел был выбран линейный 
конгруэнтный метод. Вычисление последовательности случайных чисел происходит 
следующим образом (\ref{eq:cong}):
\begin{equation}\label{eq:cong}
    X_{n+1} = (a \cdot X_n + c)\;mod\;m,
\end{equation}
где  $X_{n+1}$ -- это следующее число в последовательности, $a$ – множитель, 
причём $0 \leq a \leq m$, $c$ – приращение, причём $0 \leq c \leq m$, $m$ -- натуральное число, 
относительно которого вычисляется остаток от деления, причём $m \geq 2$.

При выборе значения $m$ необходимо учитывать следующие условия:
\begin{itemize}
    \item данное число должно быть довольно большим, так как период не
     может иметь более $m$ элементов;
    \item данное значение должно быть таким, чтобы случайные значения вычислялись быстро.
\end{itemize}
В качестве констант в данной работе используются следующие значения: $m = 36261,\;a = 66037,\;c = 312500$.

Для реализации табличного способа заранее подготовлено три таблицы с одноразрядными, двухразрядными и трехразрядными числами.

Для проверки случайности был использован собственный критерий, который формируется следующим 
способом:

\noindentПусть на вход алгоритму представлена последовательность \eqref{eq:seq}:

\begin{equation}
    a = a_1, a_2, a_3, \cdots, a_n.
\end{equation}\label{eq:seq}
С помощью этой последовательности вычисляется следующая последовательность \ref{eq:deltaseq}:

\begin{equation}\label{eq:deltaseq}
    \Delta = \Delta_1, \Delta_2, \Delta_3, \cdots, \Delta_m,\; \text{где}\; \Delta_i = a_{i + 1} - a_{i},\;m=n-1.
\end{equation}

Для последовательности $\Delta$ вычисляется коэффициент вариации \ref{eq:var}:
\begin{equation}\label{eq:var}
    V = \cfrac{\sigma}{\bar{\Delta}},
\end{equation}
где дисперсия вычисляется согласно \ref{eq:sig}:
\begin{equation}\label{eq:sig}
        \sigma=\sqrt{\frac{1}{m}\sum_{i=1}^m\left(\Delta_i-\bar{\Delta}\right)^2},
\end{equation}
а выборочное среднее согласно \ref{sample}:
\begin{equation}\label{sample}
    \displaystyle {\overline {\Delta}}={\frac {1}{m}}\sum \limits _{i=1}^{m}\Delta_{i}.
\end{equation}