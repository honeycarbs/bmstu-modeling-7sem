\chapter{Программный код}

Для реализации программы был выбран язык Python. % :(


\noindentНа листинге \ref{lst:cong} представлена реализация линейного конгруэнтного метода.
\begin{lstlisting}[language=Python,
    frame= tb,
    numbers=left,
    numberstyle=\footnotesize,
    caption={Релаизация линейного конгруэнтного метода},
    label={lst:cong}]
class Congruential():
    def __init__(self):
        self.cur = 10
        self.m = 36261
        self.a = 66037
        self.c = 312500

    def Generate(self, lo=0, hi=9, n=500):
        res = []
        for i in range(n):
            self.cur = (self.a * self.cur + self.c) % self.m
            num = int(lo + self.cur % (hi - lo))
            res.append(num)
        return res
\end{lstlisting}
На листинге \ref{lst:tab} представлена реализация табличного метода. 
\begin{lstlisting}[language=Python,
    frame= tb,
    numbers=left,
    numberstyle=\footnotesize,
    caption={Релаизация табличного метода},
    label={lst:tab}]
from pathlib import Path

class BuiltinTable():
    def __init__(self):
        self.dataFolder = Path("rand/meta/")
        self.oneDigitFilename = 'onedigit.txt'
        self.twoDigitsFilename = 'twodigits.txt'
        self.threeDigitsFilename =  "threedigits.txt"
    def Generate(self, digits = 1):
        if digits == 1:
            fileToOpen = self.dataFolder / self.oneDigitFilename
        elif digits == 2:
            fileToOpen = self.dataFolder / self.twoDigitsFilename
        else:
            fileToOpen = self.dataFolder / self.threeDigitsFilename
        
        f = open(fileToOpen, 'r')

        nums = []
        for line in f:
            nums.append(int(line.strip('\n')))
            
        return nums
\end{lstlisting}

На листинге \ref{lst:cri} представлена реализация собственного критерия проверки случайности.
\begin{lstlisting}[language=Python,
    frame= tb,
    numbers=left,
    numberstyle=\footnotesize,
    caption={Реализация собственного критерия проверки случайности},
    label={lst:cri}]
def getDeltas(a):
    deltas = []
    for i in range(len(a) - 1):
        deltas.append(abs(a[i + 1] - a[i]))
    return deltas

def variance(data):
    n = len(data)
    mean = sum(data) / n
    deviations = [(x - mean) ** 2 for x in data]
    variance = sum(deviations) / n

return variance

def randomnessTest(a):
    deltas = getDeltas(a)
    n = len(deltas)
    sample = 1 / n * sum(deltas)

    if sample == 0:
    return 0.0

    Vkoef = math.sqrt(variance(deltas)) / sample
    return Vkoef 
\end{lstlisting}