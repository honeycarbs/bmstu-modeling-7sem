\chapter{Программный код}

Для реализации программы был выбран язык Python. % :(

\noindent На листинге \ref{lst:dt} представлена реализация пошагового метода.
\begin{lstlisting}[language=Python,
    frame= tb,
    numbers=left,
    numberstyle=\footnotesize,
    caption={Релаизация пошагового метода},
    label={lst:dt}]
def TimekeepingSimulation(distribution, jobProcessor, jobsAmount=0, returnPercentage=0, step=0.001):
    processedJobs = 0
    currentTime = step
    generationTime = distribution.GetRandomValue()
    processTime = 0
    currentQueueLen = maxQueueLen = 0
    generatedJobs = 0
    returnedJobs = 0

    while processedJobs < jobsAmount + returnedJobs:

        if currentTime > generationTime and generatedJobs <= jobsAmount:
            currentQueueLen += 1
            generatedJobs += 1
            if currentQueueLen > maxQueueLen:
                maxQueueLen = currentQueueLen
                generationTime += distribution.GetRandomValue()
        if currentTime > processTime:
            if currentQueueLen > 0:
            processedJobs += 1

            if random.randint(1, 100) <= returnPercentage:
                returnedJobs += 1
                currentQueueLen += 1

            currentQueueLen -= 1
            processTime += jobProcessor.GetRandomValue()
        currentTime += step
    return maxQueueLen, processedJobs, returnedJobs
\end{lstlisting}
На листинге \ref{lst:ev} представлена реализация событийного метода. 
\begin{lstlisting}[language=Python,
    frame= tb,
    numbers=left,
    numberstyle=\footnotesize,
    caption={Релаизация событийного метода},
    label={lst:ev}]
def EventSimulation(distribution, jobProcessor, jobsAmount=0, returnPercentage=0):
    processedJobs = 0
    currentQueueLen = 0
    maxQueueLen = 0

    jobs = [[distribution.GetRandomValue(), 'g']]

    free, isProcessed = True, False

    generatedJobs = 0
    returnedJobs = 0

    while processedJobs < jobsAmount + returnedJobs:

        job = jobs.pop(0)

        if job[1] == 'g' and generatedJobs <= jobsAmount:

            currentQueueLen += 1
            generatedJobs += 1

            if currentQueueLen > maxQueueLen:
                maxQueueLen = currentQueueLen

            enqueueJob(jobs, [job[0] + distribution.GetRandomValue(), 'g'])

            if free:
                isProcessed = True

        elif job[1] == 'p':

            processedJobs += 1

            if random.randint(1, 100) <= returnPercentage:
                returnedJobs += 1
                currentQueueLen += 1

            isProcessed = True

        if isProcessed:

            if currentQueueLen > 0:
                currentQueueLen -= 1
                t = jobProcessor.GetRandomValue()
                enqueueJob(jobs, [job[0] + t, 'p'])
                free = False
            else:
                free = True
            isProcessed = False

    return maxQueueLen, processedJobs, returnedJobs
\end{lstlisting}

На листинге \ref{lst:enc} представлена функция помещения заявки в буфер.
\begin{lstlisting}[language=Python,
    frame= tb,
    numbers=left,
    numberstyle=\footnotesize,
    caption={Вспомогательная функция помещения заявки в буфер},
    label={lst:enc}]
def enqueueJob(jobs, job):
    i = 0
    while i < len(jobs) and jobs[i][0] < job[0]:
        i += 1
    if 0 < i < len(jobs):
        jobs.insert(i - 1, job)
    else:
        jobs.insert(i, job)
\end{lstlisting}


На листинге \ref{lst:dist} представлены функции распределений: равномерного и нормального.

\newpage
\begin{lstlisting}[language=Python,
    frame= tb,
    numbers=left,
    numberstyle=\footnotesize,
    caption={Функции распределений},
    label={lst:dist}]
class UniformDistributon:
    def __init__(self, lo: float, hi: float):
        self.lo = lo
        self.hi = hi
    
def GetRandomValue(self):
    return self.lo + (self.hi - self.lo) * random.random()

    class GaussianDistributon:
    def __init__(self, mu: float, sigma: float):
        self.mu = mu
        self.sigma = sigma
    
    def GetRandomValue(self):
        return normal(self.mu, self.sigma)
\end{lstlisting}