\usepackage[T2A, T1]{fontenc}
\usepackage[utf8]{inputenc}
\usepackage[english,main=russian]{babel}

\usepackage{fix-cm}

%\usepackage{lmodern}

\usepackage{microtype}
\microtypesetup{expansion=false}

\usepackage[
	left=30mm,
	right=10mm, 
	top=20mm,
	bottom=20mm,
]{geometry}

\usepackage{microtype} % Настройка переносов
\sloppy

\usepackage{setspace} % Настройка межстрочного интервала
\onehalfspacing

\usepackage{indentfirst} % Настройка абзацного отступа
\setlength{\parindent}{12.5mm}

\usepackage[unicode,hidelinks]{hyperref}
\usepackage{xifthen}

\usepackage[normalem]{ulem}
% Текст под линией 
\newcommand*{\undertext}[2]{%
	\begin{tabular}[t]{@{}c@{}}%
		#1\\\relax(\scriptsize#2)%
	\end{tabular}
}

% горизонтальная линия
\makeatletter
\newcommand{\vhrulefill}[1]
{
	\leavevmode\leaders\hrule\@height#1\hfill \kern\z@
}

% Настройка заголовков
% \makeatletter
% \renewcommand\LARGE{\@setfontsize\LARGE{22pt}{20}}
% \renewcommand\Large{\@setfontsize\Large{20pt}{20}}
% \renewcommand\large{\@setfontsize\large{16pt}{20}}
% \makeatother

\usepackage{titlesec}

\usepackage{titletoc}

\titlecontents{chapter}[0pt]
{}
{\large\thecontentslabel. \enspace}
{}
{\titlerule*[1pc]{}\contentspage}

\newcommand{\makechapterdots}{
	\titlecontents{chapter}[0pt]
	{}
	{\large\thecontentslabel. \enspace}
	{}
	{\titlerule*[1pc]{.}\contentspage}
}


\titlecontents{section}[20pt]
{}
{\large\thecontentslabel. \enspace}
{}
{\titlerule*[1pc]{.}\contentspage}

\titlecontents{subection}[20pt]
{}
{\large\thecontentslabel. \enspace}
{}
{\titlerule*[1pc]{.}\contentspage}


\titleformat{\chapter}[block]
{\bfseries\large\scshape}{\thechapter.}{0.5em}{\Large\scshape}

%\titleformat{name=\chapter,numberless}[block]
%{\hspace{\parindent}}{}{0pt}{\large\scshape\centering}

\titleformat{\section}[block]
{\bfseries\large\scshape}{\thesection}{0.5em}{\large\scshape\raggedright}

\titleformat{\subsection}[block]
{\bfseries\hspace{\parindent}\normalsize\scshape}{\thesubsection}{0.5em}{\large\scshape\raggedright}

\titleformat{\subsubsection}[block]
{\bfseries\large\scshape}{\thesubsection}{0.5em}{\large\scshape\raggedright}

\titlespacing{\chapter}{12.5mm}{-22pt}{10pt}
\titlespacing{\section}{12.5mm}{10pt}{10pt}
\titlespacing{\subsection}{12.5mm}{10pt}{10pt}
\titlespacing{\subsubsection}{12.5mm}{10pt}{10pt}

\def\labelitemi{$\circ$}


% ---------------------------------------- CAPTION --------------------------------- %

\usepackage[
labelsep=endash,
singlelinecheck=false,
]{caption}

\captionsetup[figure]{justification=centering}
\captionsetup[table]{justification=raggedleft}
\captionsetup[listing]{justification=raggedright}


% ---------------------------------------- TABLE  ----------------------------------------

\usepackage{xcolor}
\usepackage{tabularx}
\usepackage{booktabs}
\usepackage{multirow}

% ---------------------------------------- FIGURE ----------------------------------------

\usepackage{graphicx}
\usepackage{float}
\usepackage{wrapfig}
\usepackage{tikzscale}
\usepackage{svg}
\usepackage{subcaption}

\usepackage{pgfplots}
\pgfplotsset{compat=newest}

% ----------------------------------------- MATH -----------------------------------------

\RequirePackage{lscape}
\RequirePackage{afterpage}

\RequirePackage{amsmath}
\RequirePackage{amssymb}

% ----------------------------------------- LISTINGS -----------------------------------------

\RequirePackage{listings}
\RequirePackage{listingsutf8}
\lstset{
	inputencoding=utf8/koi8-r,
	basicstyle=\small\ttfamily,
	rulecolor=\color{black},
	escapeinside={\%*}{*)},
	breaklines=true,
	breakatwhitespace=true,
	tabsize=4,
	showstringspaces=false,
	float=h!,
	abovecaptionskip=-5pt,
}

\makeatletter
\long\def\@makecaption#1#2{%
	\vskip\abovecaptionskip
	#1: #2 \\ \par
	\vskip\belowcaptionskip}%
\makeatother


% ----------------------------------------- BIBLIO ---------------------------------------

\usepackage[
	style=gost-numeric,
	language=auto,
	autolang=other,
	sorting=none,
]{biblatex}
\usepackage{csquotes}
\DeclareFieldFormat{urldate}{(дата обращения:\addspace\thefield{urlday}\adddot \thefield{urlmonth}\adddot\thefield{urlyear})}