\chapter{Программный код}

Для реализации программы был выбран язык Python. % :(

На листинге \ref{lst:gen} представлена генерация числа из заданного интервала.
\begin{lstlisting}[language=Python,
    frame= tb,
    numbers=left,
    numberstyle=\footnotesize,
    caption={Генерация числа из заданного интервала},
    label={lst:gen}]
class Generator:
    def __init__(self, base, err):
        self.low = base - err
        self.high = base + err

    def generationTime(self):
        return random.randint(self.low, self.high)
\end{lstlisting}

На листинге \ref{lst:op} представлена реализация оператора.
\begin{lstlisting}[language=Python,
	frame= tb,
	numbers=left,
	numberstyle=\footnotesize,
	caption={Реализация оператора},
	label={lst:op}]
class Operator(Generator):
    def __init__(self, base, err):
        super().__init__(base, err)
        self.busy = False
\end{lstlisting}

На листинге \ref{lst:comp} представлена реализация компьютера.
\begin{lstlisting}[language=Python,
	frame= tb,
	numbers=left,
	numberstyle=\footnotesize,
	caption={Реализация компьютера},
	label={lst:comp}]
class Computer:
    def __init__(self, time):
        self.generationTime = time
        self.busy = False
\end{lstlisting}

На листинге \ref{lst:qu} представлена реализация СМО.
\begin{lstlisting}[language=Python,
	frame= tb,
	numbers=left,
	numberstyle=\footnotesize,
	caption={Реализация СМО},
	label={lst:qu}]
class QueueNetwork:
    def __init__(self, n):
        self.n = n
        self.jobs = list()
        self.clients = Generator(10, 2)
        self.operators = [Operator(20, 5), Operator(40, 10), Operator(40, 20)]
        self.computers = [Computer(15), Computer(30)]
        self.lines = [0, 0]
        self.processed = 0
        self.lost = 0

    def reset(self, n):
        self.n = n
        self.jobs = list()
        self.clients = Generator(10, 2)
        self.operators = [Operator(20, 5), Operator(40, 10), Operator(40, 20)]
        self.computers = [Computer(15), Computer(30)]
        self.lines = [0, 0]
        self.processed = 0
        self.lost = 0

    def model(self):
        self.AddJob([self.clients.generationTime(), 'client'])
        while self.processed + self.lost < self.n:
            job = self.jobs.pop(0)
            if job[1] == 'client':
                self.newClient(job[0])
            elif job[1] == 'operator':
                self.OperatorFinished(job)
            elif job[1] == 'pc':
                self.ComputerFinished(job)
        return self.lost

    def newClient(self, time):
        i = 0
        while i < 3 and self.operators[i].busy:
            i += 1
        if i == 3:
            self.lost += 1
        else:
            self.operators[i].busy = True
            self.AddJob(
                [time + self.operators[i].generationTime(), 'operator', i])
        self.AddJob([time + self.clients.generationTime(), 'client'])

    def OperatorFinished(self, job):
        self.operators[job[2]].busy = False
        if job[2] < 2:
            self.lines[0] += 1
            job[2] = 0
        else:
            self.lines[1] += 1
            job[2] = 1
        self.ComputerGetJob(job)

    def ComputerGetJob(self, job):
        cnum = job[2]
        if not self.computers[cnum].busy and self.lines[cnum] > 0:
            self.computers[cnum].busy = True
            self.AddJob(
                [job[0] + self.computers[cnum].generationTime, 'pc', cnum])
            self.lines[cnum] -= 1

    def ComputerFinished(self, job):
        self.computers[job[2]].busy = False
        self.processed += 1
        self.ComputerGetJob(job)

    def AddJob(self, job: list):
        i = 0
        while i < len(self.jobs) and self.jobs[i][0] < job[0]:
            i += 1
        self.jobs.insert(i, job)
	
\end{lstlisting}
