\chapter{Задание}

На некоторой неизвестной лучшей кафедре в в некотором неизвестном лучшем техническом вузе преподаватель принимает лабораторные работы. Студентам осталось сдать ему три лабораторные работы:
\begin{itemize}
	\item лабораторная работа <<\textit{ассиметричное шифрование}>> (под номером 4) сдается~$6\pm2$~мин.;
	\item лабораторная работа <<\textit{электронная подпись}>> (под номером 5) сдается~$4\pm1$~мин.;
	\item лабораторная работа <<\textit{сжатие}>> (под номером 6) сдается~$10\pm5$~мин.
\end{itemize}
Лабораторные работы сдаются по порядку. Если студент плохо подготовился, то он не может продолжить сдавать следующую лабораторную работу, даже если она у студента готова. 


Возможности студентов в день сдачи представлены в таблице \ref{tab:stud-op}.

\newcolumntype{C}{>{\centering\arraybackslash}p{0.3\textwidth}}
\begin{table}[H]
	\caption{Возможности студентов в день сдачи}
	\label{tab:stud-op}
	\centering
	\begin{tabular}{|c|C|C|}
		\hline
		№ & время, мин.  & описание \\ \hline
		1 & 6$\pm$2  & сдать 4-ю ЛР \\ \hline
		2 & 4$\pm$1  & сдать 5-ю ЛР \\ \hline
		3 & 10$\pm$5 & сдать 6-ю ЛР \\ \hline
		4 & 10$\pm$3 & сдать 4-ю и 5-ю ЛР \\ \hline
		5 & 14$\pm$6 & сдать 5-ю и 6-ю ЛР \\ \hline
		6 & 20$\pm$8 & закрыть курс \\ \hline
	\end{tabular}
\end{table}
Преподаватель принимает принимает лабораторные 4,25 часа. Поток студентов у преподавателя постоянный.


Осталось 3 сдачи до Нового Года. Сколько студентов закроют лабораторные работы по курсу до Нового Года, а сколько студентов будут плакать горькими слезами?

\chapter{Концептуальная модель}
Система состоит из двух блоков:
\begin{itemize}
	\item \textbf{блок имитатора воздействия внешней среды}:\\
	Система имеет шесть генераторов (по одному на каждую возможность студента). 	Чтобы поток студентов был постоянным, заявки генерируются с частотой равной минимальному времени сдачи лабораторной работы.
	\item \textbf{блок функций системы:}\\
	Система имеет шесть функционирующих блоков -- преподаватель принимает студентов согласно возможностям в таблице \ref{tab:stud-op}. Каждая сдача в среднем занимает занимает преподавателя на время, приведенное выше в таблице.
\end{itemize}

Преподаватель принимает в день сдачи 4 часа 15 минут, то есть~$4,5\cdot60=255$~минут. Необходимо смоделировать 3 дня сдачи -- 3 периода от 0 до 255 единиц времени.

